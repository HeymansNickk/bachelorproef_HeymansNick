%%=============================================================================
%% Voorwoord
%%=============================================================================

\chapter*{\IfLanguageName{dutch}{Woord vooraf}{Preface}}
\label{ch:voorwoord}

%% TODO:
%% Het voorwoord is het enige deel van de bachelorproef waar je vanuit je
%% eigen standpunt (``ik-vorm'') mag schrijven. Je kan hier bv. motiveren
%% waarom jij het onderwerp wil bespreken.
%% Vergeet ook niet te bedanken wie je geholpen/gesteund/... heeft

Deze bachelorproef luidt het einde van mijn driejarige opleiding Toegepaste Informatica aan de HoGent in. Het onderwerp voor deze bachelorproef werd gekozen vanuit persoonlijke interesses. Hierin probeer ik aan te tonen hoe de beveiliging van een Kubernetes cluster werkt en waarom deze zo belangrijk is. Door de COVID-pandemie zijn de laatste drie semesters van de opleiding op een zeer speciale manier verlopen. Deze pandemie heeft niet enkel effect gehad op de manier van les volgen maar heeft tevens een grote impact gehad op de manier waarop er naar \textit{cybersecurity} gekeken wordt. 

Ik apprecieer alle lectoren en docenten die mij in de afgelopen drie jaar enorm geholpen en gesteund hebben. Mijn ouders zou ik willen bedanken om mij de kans te geven om verder te studeren en mij verder te steunen tijdens mijn opleiding. Bij het schrijven van deze bachelorproef heb ik van verschillende personen heel wat hulp gekregen. Als eerste wil ik mijn promotor Wim De Bruyn en mij co-promotor Steven Trescinski bedanken. Vervolgens wil ik mijn vriendin Lisa Jacquemijn bedanken voor de steun en motivatie tijdens het schrijven van deze bachelorproef.  Verder wil ik ook Robin Ophalvens bedanken voor het geven van zijn zeer uitgebreide feedback.