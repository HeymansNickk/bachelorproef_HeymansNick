%%=============================================================================
%% Conclusie
%%=============================================================================

\chapter{Conclusie}
\label{ch:conclusie}

% TODO: Trek een duidelijke conclusie, in de vorm van een antwoord op de
% onderzoeksvra(a)g(en). Wat was jouw bijdrage aan het onderzoeksdomein en
% hoe biedt dit meerwaarde aan het vakgebied/doelgroep?
% Reflecteer kritisch over het resultaat. In Engelse teksten wordt deze sectie
% ``Discussion'' genoemd. Had je deze uitkomst verwacht? Zijn er zaken die nog
% niet duidelijk zijn?
% Heeft het onderzoek geleid tot nieuwe vragen die uitnodigen tot verder
%onderzoek?

%Wat zijn de belangrijkste security risico's?
%	- fout opzetten cluster
%	- bugs

%Welke \textit{best practices} kunnen toegepast worden? 
%	- PODSECPOL
%	- RBAC

%Welke \textit{security tools} zijn er en hoe werken ze? 
%	- Zeer veel verschillende
%	- Kube-bench en hunter
	

%Welke impact hebben \textit{best practices} en \textit{security tools} op verschillende criteria?
Dit onderzoek heeft aangetoond dat er veel verschillende manieren zijn om een Kubernetes cluster te beveiligen. Gaande van het correct volgen van \textit{best practices} tot het gebruik van groot aantal beveiligings-tools. Er is ook gebleken dat het beveiligen van een K8s cluster heel wat werk met zich meebrengt, waardoor sommige organisaties dit als optioneel zien. Uit het onderzoek bleken er twee zeer belangrijke risico's bij het opzetten van een K8s cluster, namelijk menselijke fouten bij het configureren van de cluster en bugs die onvermijdelijk zijn in een project van deze schaal.

De \textit{best practices}, waaronder het gebruik van \textit{Pod Security Polices} en RBAC, blijken zeer krachtige tools bij het beveiligen van een cluster. Tijdens het onderzoek is gebleken dat er zeer veel \textit{best practices} zijn. Deze hadden voornamelijk betrekking tot het correct gebruiken van ingebouwde opties. Het nadeel aan deze \textit{best practices} is dat het zeer complex kan worden in een grote productieomgeving, waardoor het soms achterwege wordt gelaten. 

Omdat K8s zo veel gebruikt wordt, zijn er tevens veel bedrijven die hun eigen tools voor het beveiligen van een cluster ontwikkelen. Deze tools zijn meestal, net zoals Kubernetes zelf, open source waardoor iedereen ze kan gebruiken. De twee meest gebruikte en bekendste beveiligings-tools zijn Kube-bench en Kube-hunter. Beiden worden gebruikt om de beveiliging van de cluster te controleren, maar ze doen dit elk op hun eigen manier. Kube-bench controleert de cluster op basis van vooropgestelde benchmarks, dit terwijl Kube-hunter eerder wordt gebruikt als automatische penetratie test.

Uit het onderzoek is ook gebleken dat zowel het toepassen van \textit{best practices} als het gebruik van beveiligings-tools weinig effect heeft op de vooropgestelde criteria. De stijging in resource gebruik was detecteerbaar en verwacht maar in de meeste gevallen was deze praktisch verwaarloosbaar. Ondanks zijn er nog veel bedrijven die te weinig inzetten op de beveiliging van hun K8s clusters. In toekomstig onderzoek zou er mogelijks verder gekeken kunnen worden naar de achterliggende reden voor deze nalatigheid.  