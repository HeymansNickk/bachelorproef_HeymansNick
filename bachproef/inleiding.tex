%%=============================================================================
%% Inleiding
%%=============================================================================

\chapter{\IfLanguageName{dutch}{Inleiding}{Introduction}}
\label{ch:inleiding}

%De inleiding moet de lezer net genoeg informatie verschaffen om het onderwerp te begrijpen en in te zien waarom de onderzoeksvraag de moeite waard is om te onderzoeken. In de inleiding ga je literatuurverwijzingen beperken, zodat de tekst vlot leesbaar blijft. Je kan de inleiding verder onderverdelen in secties als dit de tekst verduidelijkt. Zaken die aan bod kunnen komen in de inleiding~\autocite{Pollefliet2011}:


%\begin{itemize}
%  \item context, achtergrond
%  \item afbakenen van het onderwerp
%  \item verantwoording van het onderwerp, methodologie
%  \item probleemstelling
%  \item onderzoeksdoelstelling
%  \item onderzoeksvraag
%  \item \ldots
%\end{itemize}
Tijdens de ontwikkeling van traditionele applicaties wordt de applicatie uitgewerkt in een specifieke testomgeving. Bij het overzetten van de applicatie van de test- naar de productieomgeving (i.e., van een Linux testomgeving naar een Windows productieomgeving), komen er vaak problemen naar boven). Deze problemen kunnen vermeden worden door gebruik te maken van containers en vergemakkelijken daarbij het uitrollen en schalen.

Een container is een pakket waar één enkele applicatie in zit, samen met alle nodige afhankelijkheden \autocite{Education2019}. Dit zorgt ervoor dat deze gemakkelijk en snel van de ene omgeving naar de andere kan overgezet worden.

Naarmate het gebruik van containers steeg, steeg ook de nood om deze vanuit één centrale locatie te beheren. Om aan deze vraag te voldoen werden container orkestratie tools, zoals Kubernetes\footnote{https://kubernetes.io/}, ontwikkeld. Deze tools helpen bij het opzetten, uitbreiden en verbinden van een grote hoeveelheid containers.

In deze bachelorproef worden de moderne veiligheidsrisico’s geanalyseerd die gepaard gaan met container virtualisatie en container orkestratie. Hierbij zal specifiek gekeken worden naar de grootste veiligheidsrisico’s, welke effecten deze kunnen hebben op een productieomgeving en hoe deze vermeden kunnen worden.

Aan alle technologieën zijn nu eenmaal veiligheidsrisico’s verbonden, container virtualisatie en container orkestratie vormen hier geen uitzondering op de regel. Een groot onderdeel van container beveiliging zijn de zogenaamde beveiligings-tools(zoals \textit{Project Calico} en \textit{Kube-hunter}).

In voorgaand onderzoek, zoals \autocite{Shamim2020}, werd er reeds gefocust op de grootste veiligheidsrisico's. Echter is er zeer weinig ingegaan op de effecten bij het toepassen van \textit{best practices} en het gebruik van  beveiligings-tools. Met deze paper tracht ik het gebruik, en de daaraan verbonden risico’s, van container virtualisatie en container orkestratie te onderzoeken. Daarnaast zal er ook gekeken worden naar hoe de verschillende beveiligings-tools kunnen helpen bij het beveiligen van containers.

Ten slotte wordt er onderzocht hoe deze risico’s vermeden of opgelost kunnen worden en welk effecten ze hebben op de relevante criteria. Dit laatste zal via een \textit{proof-of-concept} opstelling gebeuren.

\section{\IfLanguageName{dutch}{Probleemstelling}{Problem Statement}}
\label{sec:probleemstelling}

%Uit je probleemstelling moet duidelijk zijn dat je onderzoek een meerwaarde heeft voor een concrete doelgroep. De doelgroep moet goed gedefinieerd en afgelijnd zijn. Doelgroepen als ``bedrijven,'' ``KMO's,'' systeembeheerders, enz.~zijn nog te vaag. Als je een lijstje kan maken van de personen/organisaties die een meerwaarde zullen vinden in deze bachelorproef (dit is eigenlijk je steekproefkader), dan is dat een indicatie dat de doelgroep goed gedefinieerd is. Dit kan een enkel bedrijf zijn of zelfs één persoon (je co-promotor/opdrachtgever).
De probleemstelling houdt in dat veel bedrijven gebruik maken van containers en container orkestratie zonder hierbij al te veel aandacht te besteden aan de beveiliging hiervan. Daarnaast dient er gekeken te worden naar wat voor effecten de beveiliging van een container omgeving met zich mee brengt en hoe men best te werk gaat.

\section{\IfLanguageName{dutch}{Onderzoeksvraag}{Research question}}
\label{sec:onderzoeksvraag}

%Wees zo concreet mogelijk bij het formuleren van je onderzoeksvraag. Een onderzoeksvraag is trouwens iets waar nog niemand op dit moment een antwoord heeft (voor zover je kan nagaan). Het opzoeken van bestaande informatie (bv. ``welke tools bestaan er voor deze toepassing?'') is dus geen onderzoeksvraag. Je kan de onderzoeksvraag verder specifiëren in deelvragen. Bv.~als je onderzoek gaat over performantiemetingen, dan

Wat zijn de belangrijkste beveiligingsrisico's? Welke beveiligings-tools zijn er en hoe werken ze? Welke \textit{best practices} kunnen toegepast worden? Welke impact hebben \textit{best practices} en beveiligings-tools op verschillende criteria?

\section{\IfLanguageName{dutch}{Onderzoeksdoelstelling}{Research objective}}
\label{sec:onderzoeksdoelstelling}

%Wat is het beoogde resultaat van je bachelorproef? Wat zijn de criteria voor succes? Beschrijf die zo concreet mogelijk. Gaat het bv. om een proof-of-concept, een prototype, een verslag met aanbevelingen, een vergelijkende studie, enz.

Het doel van deze bachelorproef is hoofdzakkelijk om tot een verslag te komen met daarin aanbevelingen omtrent het beveiligen van een container cluster. Deze aanbevelingen zullen gestaafd worden door enkele scenarios en hun effect op enkele criteria.

\section{\IfLanguageName{dutch}{Opzet van deze bachelorproef}{Structure of this bachelor thesis}}
\label{sec:opzet-bachelorproef}

% Het is gebruikelijk aan het einde van de inleiding een overzicht te
% geven van de opbouw van de rest van de tekst. Deze sectie bevat al een aanzet
% die je kan aanvullen/aanpassen in functie van je eigen tekst.

De rest van deze bachelorproef is als volgt opgebouwd:

In Hoofdstuk~\ref{ch:stand-van-zaken} wordt een overzicht gegeven van de stand van zaken binnen het onderzoeksdomein, op basis van een literatuurstudie.

In Hoofdstuk~\ref{ch:methodologie} wordt de methodologie toegelicht en worden de gebruikte onderzoekstechnieken besproken om een antwoord te kunnen formuleren op de onderzoeksvragen.

% TODO: Vul hier aan voor je eigen hoofstukken, één of twee zinnen per hoofdstuk

In Hoofdstuk~\ref{ch:conclusie}, tenslotte, wordt de conclusie gegeven en een antwoord geformuleerd op de onderzoeksvragen. Daarbij wordt ook een aanzet gegeven voor toekomstig onderzoek binnen dit domein.