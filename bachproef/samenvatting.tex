%%=============================================================================
%% Samenvatting
%%=============================================================================

% TODO: De "abstract" of samenvatting is een kernachtige (~ 1 blz. voor een
% thesis) synthese van het document.
%
% Deze aspecten moeten zeker aan bod komen:
% - Context: waarom is dit werk belangrijk?
% - Nood: waarom moest dit onderzocht worden?
% - Taak: wat heb je precies gedaan?
% - Object: wat staat in dit document geschreven?
% - Resultaat: wat was het resultaat?
% - Conclusie: wat is/zijn de belangrijkste conclusie(s)?
% - Perspectief: blijven er nog vragen open die in de toekomst nog kunnen
%    onderzocht worden? Wat is een mogelijk vervolg voor jouw onderzoek?
%
% LET OP! Een samenvatting is GEEN voorwoord!

%%---------- Nederlandse samenvatting -----------------------------------------
%
% TODO: Als je je bachelorproef in het Engels schrijft, moet je eerst een
% Nederlandse samenvatting invoegen. Haal daarvoor onderstaande code uit
% commentaar.
% Wie zijn bachelorproef in het Nederlands schrijft, kan dit negeren, de inhoud
% wordt niet in het document ingevoegd.

\IfLanguageName{english}{%
\selectlanguage{dutch}
\chapter*{Samenvatting}
\lipsum[1-4]
\selectlanguage{english}
}{}

%%---------- Samenvatting -----------------------------------------------------
% De samenvatting in de hoofdtaal van het document

\chapter*{\IfLanguageName{dutch}{Samenvatting}{Abstract}}

In deze bachelorproef worden de moderne veiligheidsrisico’s geanalyseerd die gepaard gaan met container virtualisatie en container orkestratie. Hierbij zal specifiek gekeken worden naar de grootste veiligheidsrisico’s, welke effecten deze kunnen hebben op een productieomgeving en hoe deze vermeden kunnen worden. Tevens zal onderzocht worden welke ’security tools’ (zoals ’Project Calico’ en ’Kube-hunter’) er bestaan en hoe deze ingezet kunnen worden bij het beveiligen van een container cluster. De laatste jaren is het gebruik van containers in de IT infrastructuur fors gestegen en dit zal zo blijven evolueren in de komende jaren. Aan alle technologieen zijn nu eenmaal veiligheidsrisico’s verbonden, container orkestratie vormt hier geen uitzondering op de regel. In voorgaand onderzoek werd er reeds gefocust op de grootste veiligheidsrisico’s desondanks is er zeer weinig ingegaan op de effecten bij het toepassen van ’best practices’. Met deze paper tracht ik het gebruik en de daaraan verbonden risico’s van container virtualisatie en container orkestratie te onderzoeken. Daarnaast zal er ook gekeken worden naar hoe de verschillende ’security tools’ kunnen helpen bij het beveiligen van containers. Ten slotte wordt er onderzocht hoe deze risico’s vermeden of opgelost kunnen worden en welk effecten ze hebben op de relevante criteria. Dit laatste zal via een ’proof-of-concept’ opstelling gebeuren.
